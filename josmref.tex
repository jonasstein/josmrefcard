\documentclass[a4paper,11pt,notumble]{leaflet}

% So, jetzt bin ich hier an der Reihe ;-)

\usepackage[utf8]{inputenc}
\usepackage[ngerman]{babel}
\usepackage{ae}
\usepackage{amsmath}
\usepackage{setspace}
\usepackage{multicol}
% \usepackage{keystroke}
% ftp://dante.ctan.org/tex-archive/macros/latex/contrib/keystroke/key-test.pdf

\usepackage{graphicx}
\graphicspath{{gfx/}}

\newcommand{\Key}[2]{{\bf {\tt #1}}\quad {#2}\\[2mm]}


\begin{document} 
{\Huge JOSM Kurzreferenz}\\
{\small PDF erstellt am \today}

Es gibt in OSM nur Knoten mit einer Liste von Schlüssel-Wert-Paaren (Notation: \texttt{key = Value}).
Sie können einen Nachfolger und einen Vorgänger haben und so einen Pfad bilden.
Eine Fläche wird durch ihren Rand als geschlossener Pfad dargestellt


\section*{Werkzeuge}
\Key{s}{Auswahlwerkzeug (select)}
\Key{a}{Knotenwerkzeug fügt Knoten hinzu (add nodes)}
\Key{c}{Pfade kombinieren (combine)}
\Key{n}{neuen Knoten hinzufügen}
\Key{p}{Pfad an diesem Knoten auftrennen}
\Key{Shift + j}{Selektierte Flächen vereinigen}
\Key{Strg + z}{Rückgängig}

\section*{Hausbau}
\Key{w}{Gebäude zeichnen}
\Key{x}{Fläche zwischen zwei Knoten wie eine Schublade ausziehen}
\Key{Shift + t}{markierte Fläche in Reihenhäuser mit Adressen aufteilen}
\Key{Shift + r}{Reihenfolge der Hausnummern von Reihenhäusern umkehren}

\section*{Kommunikation}
\Key{Strg + i}{Ausgewähltes Objekt im Browser öffnen. Etwa um Link auf ein Objekt zu mailen}

\section*{Tracks/Fotos}
Tipp: Prune kann .gpx Tracks beschneiden. 


\newpage
\section*{Wichtige Tags}
\section*{Besonders wichtige Plugins}
\begin{description}
\item[Openstreetbugs] Kartenfehler markieren und die Markierungen bearbeiten
\item[terracer] Hausbautool für Reihenhäuser
\item[validator] Prüft Karte auf logische Unstimmigkeiten
(Achtung! Erzeugt manchmal Fehlalarm!) 
\item[wms] Zeigt Luftbilder im Hintergrund an. Manchmal sehr kompliziert in der Installation, langsamer Bildaufbau. Anleitung im OSM-Wiki beachten.
\end{description}

\newpage
Diese Befehlsreferenz wurde von Jonas Stein in \LaTeX gesetzt.
Verbesserungsvorschläge bitte an news@jonasstein.de mailen.

\end{document}


%%% Local Variables: 
%%% mode: latex
%%% TeX-master: t
%%% End: 
