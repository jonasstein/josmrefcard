\documentclass[a4paper,11pt,notumble]{leaflet}
\usepackage[utf8]{inputenc}
\usepackage[T1]{fontenc}
\usepackage[ngerman]{babel}
%\usepackage{setspace}
\usepackage{hyperref}


\usepackage{graphicx}
\graphicspath{{gfx/}}

\newcommand{\Key}[2]{{\bf {\tt #1}}\quad {#2}\\[2mm]}
\newcommand{\Icon}[1]{\includegraphics[width=6mm]{#1}}


\begin{document} 
\hypersetup{
  pdftitle={JOSM Kurzreferenz},  
  pdfsubject={Kurzreferenz mit Tipps zu JOSM},
  pdfauthor={Jonas Stein, et al.}
  pdfnewwindow=true
}

{\Huge JOSM Kurzreferenz}\\
{\small PDF erstellt am \today}

Es gibt in OSM nur Knoten mit einer Liste von Schlüssel"=Wert"=Paaren
(Notation: \texttt{key\,=\,value}).  Diese können einen Nachfolger und
einen Vorgänger haben und so einen Pfad bilden.  Eine Fläche wird
durch ihren Rand als geschlossener Pfad dargestellt

\section*{JOSM Installation} 
Als aktiver Mapper benötigt man einen OSM"=Wikiaccount und einen
OSM"=API"=Account; es ist praktisch, beiden den gleichen Benutzernamen
zu geben.  Nachdem Sun Java 6 (auf Version achten!) installiert ist,
lädt man JOSM als .jar Datei herunter und startet sie etwa mit
\texttt{java -jar josm.jar}\\
Jetzt werden Benutzername und Passwort eingetragen und fehlende
Plugins installiert.


\begin{flushleft}
\section*{Besonders wichtige Plugins}
\begin{description}
\item[\Icon{openstreetbugs.png} openstreetbugs] Kartenfehler markieren
  und die Markierungen bearbeiten
\item[\Icon{terrace.png} terracer] Werkzeug für Reihenhäuser
\item[\Icon{validator.png} validator] Prüft die Daten auf logische
  Unstimmigkeiten (Achtung: Oft Fehlalarme!)
\item[\Icon{wms.png} wms] Zeigt Luftbilder im Hintergrund an. Manchmal
  kompliziert in der Installation, langsamer Bildaufbau. Anleitung im
  OSM"=Wiki beachten.
\item[\Icon{editgpxmode.png} editgpx] Ermöglicht das Löschen von
  Punkten aus GPX"=Tracks direkt in JOSM (z.B. um Punktewolken zu
  entfernen)
\item[\Icon{turnrestrictions.png} turnrestrictions] vereinfacht das
  Erstellen von Abbiegebeschränkungen
\item[\Icon{geotagging.png} photo geotagging] Speichert in die .jpg
  Datei (EXIF) permanent wo das Bild aufgenommen
  wurde. ("`Georeferenzieren"')
\end{description}

\section*{Daten holen/senden}
\Icon{download.png} \Key{Strg + Shift + d}{Daten vom OSM"=Server holen (download)}

\Icon{open.png} \Key{Strg + O}{GPS"=Track oder OSM"=Daten aus Datei einlesen (open)}
\Icon{upload.png} \Key{Strg + Shift + u}{Änderungen an OSM"=Server senden (upload)}


\section*{Werkzeuge}
\Icon{move.png} \Key{s}{Auswahlwerkzeug (select)}
\Icon{addnode.png} \Key{a}{Knoten hinzufügen (add nodes)}
\Icon{combineway.png} \Key{c}{Pfade verbinden (combine)}
\Icon{addnode.png} \Key{n}{neuen Knoten hinzufügen}
\Icon{splitway.png} \Key{p}{Pfad an diesem Knoten auftrennen}
\Icon{joinareas.png} \Key{Shift + j}{Selektierte Flächen vereinigen}
\Icon{undo.png} \Key{Strg + z}{Rückgängig}
\Icon{add.png} \Key{Alt + b}{Tag hinzufügen}
\Icon{copy.png} \Key{Strg + c}{ausgewähltes Element kopieren}
\Icon{paste.png} \Key{Strg + v}{zuvor kopierte Elemente einfügen}
\Icon{pastetags.png} \Key{Strg + Shift + v}{Tags des kopierten Objekts
  auf anderes Objekt übertragen}

%e\\[-48pt]
\section*{Hausbau}
\Icon{building.png} \Key{w}{Gebäude zeichnen}
\Icon{extrude.png} \Key{x}{Fläche zwischen zwei Knoten wie eine
  Schublade ausziehen (extrudieren)}
\Icon{terrace.png} \Key{Shift + t}{markierte Fläche in Reihenhäuser
  mit Adressen aufteilen}
\Icon{reverseterrace.png} \Key{Shift + r}{Reihenfolge der Hausnummern
  von Reihenhäusern umkehren}

\end{flushleft}

\section*{Kommunikation}
\Icon{about.png} \Key{Strg + i}{Ausgewähltes Objekt im Browser
  öffnen. Diesen eindeutigen Link kann man anderen senden um ein
  Objekt zu zeigen}
%\newpage
\section*{Tracks/Fotos}

\subsection*{Bilder georeferenzieren}
Ein Foto kann Geoinformationen (Ort der Aufnahme) enthalten.  Bei
Digitalkameras ohne GPS"=Empfänger wird ein mit GPS aufgezeichneter
Track nachträglich mit den Fotos kombiniert.

Dazu fotografiert man \textbf{während} der Tour die Uhrzeit auf dem
Display des GPS"=Gerätes. In JOSM kann beim Importieren der Bilder
dann die Uhrzeit von diesem Bild abgelesen werden. JOSM berechnet den
Zeitversatz zur Kamerauhr, korrigiert diese Zeit und ordnet dann die
Fotos den jeweiligen Koordinaten zu.

\vspace{0.5cm}

\subsection*{.gpx Track beschneiden} kann \textbf{Prune} \\
\href{http://activityworkshop.net/software/prune/}{activityworkshop.net/software/prune/}

\vspace{0.5cm}
%\newpage
\section*{Wichtige Tags}
\begin{flushleft}
\begin{description}
\item[name=*] Name des Objektes
\item[note=*] Notiz für andere Mapper
\item[fixme=*] wie \texttt{note} für unfertige Einträge. 
  Beispiel: "`fixme=Bitte Weg fortsetzen"'
\item[source=*] Datenquelle (Yahoo, GPS, survey, knowledge)
\subsection*{Straßen}
\item[highway=residential] innerörtliche Straße
\item[highway=living\_street] Spielstraße
\item[maxspeed=*] Geschwindigkeitsbegrenzung 
%zusatz km/h würde ich weglassen, da in D so üblich.
\item[lit=yes\textbar no] beleuchtet?
\subsection*{Hausnummern}
\item[addr:street=*] Straßenname
\item[addr:housenumber=*] Hausnummer 
%\\ \textit{evtl. zusammen mit}
%\item[addr:interpolation=all\textbar even\textbar odd] Hausnummern über Pfad interpolieren
%vielleicht nicht unbedingt für Anfänger, da eigentlich nur als Notlösung gedacht ~MA
\item[addr:postcode=*] Postleitzahl
\item[addr:city=*] Stadt
\item[addr:country=*] ISO"=Länderkürzel (z.B. DE)
% was anderes als DE wird der Einsteiger erstmal nicht brauchen.
\subsection*{ÖPNV}
\item[highway=bus\_stop] Bushaltestelle
\item[amenity=bus\_station] Busbahnhof
\item[railway=station] Bahnhof
\item[railway=halt] Bahn"=Haltepunkt
\subsection*{Post}
\item[amenity=post\_box] Briefkasten
\item[amenity=post\_office] Postfiliale
\item[amenity=vending\_machine] zusammen mit:
\item[vending=parcel\_pickup; parcel\_mail\_in] Packstation
\subsection*{Stadtmöblierung}
\item[amenity=waste\_basket] Mülleimer
\item[amenity=bench] Sitzbank
\item[tourism=picnic\_site] Picknick"=Platz
\item[barrier=bollard] Poller
\item[barrier=cycle\_barrier] Umlaufgitter
\item[tourism=artwork] Statue, öff. Kunstwerk
\item[amenity=telephone] öffentliches Telefon
\subsection*{Essen/Trinken}
\item[amenity=restaurant] Restaurant
\item[amenity=cafe] Café
\item[amenity=bar] Bar
\item[amenity=nightclub] Disko
\item[cuisine=*] Art der Küche
\item[smoking=yes\textbar no] Raucher\,/\,Nichtraucher
\subsection*{Sonstiges}
\item[amenity=bank] Bank (Geldinstitut)
\item[operator=*] Name des Geldinstituts
\item[amenity=atm] Geldautomat
\item[amenity=library] Bücherei
\item[amenity=kindergarten] Kindergarten
\item[amenity=school] Schule
\item[tourism=museum] Museum
\item[leisure=playground] Spielplatz
% \item[highway=stop] STOP"=Schild
% ^^ hier gibt es verschiedene Auffassungen, ob/wie das zu mappen ist. besser nicht fuer Anfaenger. ~JS
\item[shop=supermarket] Supermarkt (Liste weiterer Shops im Wiki)
\\
\item[Tipp:] Eine Übersicht wichtiger Tags steht im OSM"=Wiki unter
\href{http://wiki.osm.org/wiki/DE:Howto_Map_A}{DE:Howto\_Map\_A} und
\href{http://wiki.openstreetmap.org/wiki/DE:Map_Features}{DE:Map\_Features}
\end{description}
\end{flushleft}

%\newpage



Diese Befehlsreferenz wurde von Jonas Stein in \LaTeX{} gesetzt.
Verbesserungsvorschläge bitte an
\href{mailto:news@jonasstein.de}{news@jonasstein.de} mailen.
Die Quelldaten sind online erreichbar unter:
\href{http://github.com/jonasstein/josmrefcard/}{github.com/jonasstein/josmrefcard/}

\textbf{Dank für das Mitwirken an:} Max Andre, 
Edbert van Eimeren, Matthias Merz
\end{document}

%%% Local Variables: 
%%% mode: latex
%%% TeX-master: t
%%% End: 
