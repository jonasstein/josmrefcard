\documentclass[a4paper,11pt,notumble]{leaflet}

\usepackage[utf8]{inputenc}
\usepackage[ngerman]{babel}
%\usepackage{ae}
%\usepackage{amsmath}
%\usepackage{setspace}
%\usepackage{multicol}
\usepackage{hyperref}
% \usepackage{keystroke}
% ftp://dante.ctan.org/tex-archive/macros/latex/contrib/keystroke/key-test.pdf

\usepackage{graphicx}
\graphicspath{{gfx/}}

\newcommand{\Key}[2]{{\bf {\tt #1}}\quad {#2}\\[2mm]}
\newcommand{\Icon}[1]{\includegraphics[width=7mm]{#1}}


\begin{document} 
\hypersetup{
  pdftitle={JOSM Kurzreferenz},  
  pdfsubject={Kurzreferenz mit Tipps zu JOSM},
  pdfauthor={Jonas Stein, et al.}
  pdfnewwindow=true
}

{\Huge JOSM Kurzreferenz}\\
{\small PDF erstellt am \today}

Es gibt in OSM nur Knoten mit einer Liste von Schlüssel-Wert-Paaren (Notation: \texttt{key = Value}).
Sie können einen Nachfolger und einen Vorgänger haben und so einen Pfad bilden.
Eine Fläche wird durch ihren Rand als geschlossener Pfad dargestellt

\section*{Daten holen/senden}
\Icon{download.png} \Key{Strg + Shift + d}{Kartendaten vom OSM-Server holen (download)}
\Icon{upload.png} \Key{Strg + Shift + u}{Änderungen an OSM-Server senden (upload)}

\section*{Werkzeuge}
\Icon{move.png} \Key{s}{Auswahlwerkzeug (select)}
\Icon{addnode.png} \Key{a}{Knoten hinzufügen (add nodes)}
\Icon{combineway.png} \Key{c}{Pfade kombinieren (combine)}
\Icon{addnode.png} \Key{n}{neuen Knoten hinzufügen}
\Icon{splitway.png} \Key{p}{Pfad an diesem Knoten auftrennen}
\Icon{joinareas.png} \Key{Shift + j}{Selektierte Flächen vereinigen}
\Icon{undo.png} \Key{Strg + z}{Rückgängig}
\Icon{add.png} \Key{Strg + b}{Tag hinzufügen}
\\[-48pt]
\section*{Hausbau}
\Icon{building.png} \Key{w}{Gebäude zeichnen}
\Icon{extrude.png} \Key{x}{Fläche zwischen zwei Knoten wie eine Schublade ausziehen (extrudieren)}
\Icon{terrace.png} \Key{Shift + t}{markierte Fläche in Reihenhäuser mit Adressen aufteilen}
\Icon{reverseterrace.png} \Key{Shift + r}{Reihenfolge der Hausnummern von Reihenhäusern umkehren}
\section*{Kommunikation}
\Key{Strg + i}{Ausgewähltes Objekt im Browser öffnen. Diesen eindeutigen Link kann man anderen senden um ein Objekt zu zeigen}

\section*{Tracks/Fotos}

\subsection*{Bilder georeferenzieren} Es ist möglich, in Bildern Geoinformationen (Ort an dem das Bild erstellt wurde) zu 
speichern. Da die allermeisten Digitalkameras (noch) keinen GPS-Empfänger haben, müssen die Bilder nachträglich mit 
den Geoinformationen versehen werden. Dazu nimmt man die Uhrzeit, an der ein Bild aufgenommen wurde und sucht aus dem 
passenden GPX-Track die zugehörige Koordinate. Damit das zuverlässig klappt, muss man die Uhrzeit der Kamera, mit der 
GPS-Zeit abgleichen. \\
Dazu fotografiert man \textbf{während} der Tour das Display des GPS-Gerätes mit der eingeblendeten Uhrzeit. Später 
kann man in JOSM dann beim Importieren der Bilder die Uhrzeit von dem Bild eingeben. JOSM übernimmt dann die Zuordnung
der Fotos zu den jeweiligen Koordinaten. \\

Tipp: \textbf{Prune} kann .gpx Tracks beschneiden. 
\href{http://activityworkshop.net/software/prune/}{activityworkshop.net/software/prune/}

\newpage
\section*{Wichtige Tags}
\begin{description}
\item[name=*] Name eines Objekts
\item[note=*] Notiz für andere Mapper
\item[fixme=*] Notiz für andere Mapper, das hier ein Fehler ist
\item[source=*] Datenquelle (Yahoo, GPS, survey)
\subsection*{Hausnummern}
\item[addr:street=*] Straßenname
\item[addr:housenumber=*] Hausnummer \\ \textit{evtl. zusammen mit}
\item[addr:interpolation=all/even/odd] Hausnummern über Pfad interpolieren
\item[addr:postcode=*] Postleitzahl
\item[addr:city=*] Stadt
\item[addr:country=*] ISO-Länderkürzel (alpha2)
\subsection*{ÖPNV}
\item[highway=bus\_stop] Bushaltestelle
\item[amenity=bus\_station] Busbahnhof
\item[railway=station] Bahnhof
\item[railway=halt] Bahn-Haltepunkt
\subsection*{Post}
\item[amenity=post\_box] Briefkasten
\item[amenity=post\_office] Post'amt'
\item[amenity=vending\_machine] zusammen mit:
\item[vending=parcel\_pickup; parcel\_mail\_in] Packstation
\subsection*{Stadtmöblierung}
\item[amenity=waste\_basket] Mülleimer
\item[amenity=bench] (Park)Bank
\item[lit=yes|no] Straßenbeleuchtung
\item[tourism=picnic\_site] Picknick-Platz
\item[barrier=bollard] Poller
\item[barrier=cycle\_barrier] Umlaufgitter
\item[tourism=artwork] Statue, öffl. Kunstwerk
\item[amenity=telephone] Telefonzelle
\subsection*{Essen/Trinken}
\item[amenity=restaurant] Restaurant
\item[amenity=cafe] Café
\item[amenity=bar] Bar
\item[amenity=nightclub] Disko
\item[cuisine=*] Art der Küche
\item[smoking=yes|no] Raucher / Nichtraucher
\subsection*{Sonstiges}
\item[amenity=bank] Bank (Geldinstitut)
\item[operator=*] Name des Geldinstituts
\item[amenity=atm] Geldautomat
\item[amenity=library] Bücherei
\item[amenity=kindergarten] Kindergarten
\item[tourism=museum] Museum
\item[leisure=playground] Spielplatz
\item[highway=stop] STOP-Schild
\item[shop=supermarket] Supermarkt
\\
\item[Tipp:] Eine Übersicht über viele wichtige Tags findet sich unter: 
http://wiki.osm.org/wiki/DE:Howto\_Map\_A
\end{description}

\newpage

\section*{Besonders wichtige Plugins}
\begin{description}
\item[\Icon{openstreetbugs.png} openstreetbugs] Kartenfehler markieren und die Markierungen bearbeiten
\item[\Icon{terrace.png} terracer] Hausbautool für Reihenhäuser
\item[\Icon{validator.png} validator] Prüft Karte auf logische Unstimmigkeiten
(Achtung! Erzeugt manchmal Fehlalarme!) 
\item[\Icon{wms.png} wms] Zeigt Luftbilder im Hintergrund an. Manchmal sehr kompliziert in der Installation, langsamer Bildaufbau. 
Anleitung im OSM-Wiki beachten.
\item[\Icon{editgpxmode.png} editgpx] Ermöglicht das Löschen von Punkten aus GPX-Tracks direkt in JOSM (z.B. um Punktewolken zu entfernen)
\item[\Icon{turnrestrictions.png} turnrestrictions] vereinfacht das Erstellen von Abbiegebeschränkungen
\item[\Icon{geotagging.png} photo geotagging] Speichert in die .jpg Datei (EXIF) permanent wo das Bild aufgenommen wurde. (Georeferenzieren)
\end{description}


Diese Befehlsreferenz wurde von Jonas Stein in \LaTeX{} gesetzt.
Verbesserungsvorschläge bitte an \href{mailto:news@jonasstein.de}{news@jonasstein.de} mailen.
Die Quelldaten sind online erreichbar unter: 
\href{http://github.com/jonasstein/josmrefcard/}{github.com/jonasstein/josmrefcard/}

\textbf{Dank für das Mitwirken an:} Max Andre

\end{document}

%%% Local Variables: 
%%% mode: latex
%%% TeX-master: t
%%% End: 
