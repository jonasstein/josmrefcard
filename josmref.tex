\documentclass[a4paper,11pt,notumble]{leaflet}

\usepackage[utf8]{inputenc}
\usepackage[ngerman]{babel}
\usepackage{ae}
\usepackage{amsmath}
\usepackage{setspace}
\usepackage{multicol}
% \usepackage{keystroke}
% ftp://dante.ctan.org/tex-archive/macros/latex/contrib/keystroke/key-test.pdf

\usepackage{graphicx}
\graphicspath{{gfx/}}

\newcommand{\Key}[2]{{\bf {\tt #1}}\quad {#2}\\[2mm]}
\newcommand{\Icon}[1]{\includegraphics[width=7mm]{#1}}

\begin{document} 
{\Huge JOSM Kurzreferenz}\\
{\small PDF erstellt am \today}

Es gibt in OSM nur Knoten mit einer Liste von Schlüssel-Wert-Paaren (Notation: \texttt{key = Value}).
Sie können einen Nachfolger und einen Vorgänger haben und so einen Pfad bilden.
Eine Fläche wird durch ihren Rand als geschlossener Pfad dargestellt


\section*{Werkzeuge}
\Icon{Objekte_auswaehlen.png} \Key{s}{Auswahlwerkzeug (select)}
\Icon{Setze_Knotenpunkt.png} \Key{a}{Knotenwerkzeug fügt Knoten hinzu (add nodes)}
\Key{c}{Pfade kombinieren (combine)}
\Key{n}{neuen Knoten hinzufügen}
\Key{p}{Pfad an diesem Knoten auftrennen}
\Key{Shift + j}{Selektierte Flächen vereinigen}
\Key{Strg + z}{Rückgängig}
\Key{Strg + b}{Tag hinzufügen}

\section*{Hausbau}
\Key{w}{Gebäude zeichnen}
\Key{x}{Fläche zwischen zwei Knoten wie eine Schublade ausziehen (extrudieren)}
\Key{Shift + t}{markierte Fläche in Reihenhäuser mit Adressen aufteilen}
\Key{Shift + r}{Reihenfolge der Hausnummern von Reihenhäusern umkehren}

\section*{Kommunikation}
\Key{Strg + i}{Ausgewähltes Objekt im Browser öffnen. Etwa um Link auf ein Objekt zu mailen}

\section*{Tracks/Fotos}

\textbf{Bilder georeferenzieren} Es ist möglich, in Bildern Geoinformationen (Ort an dem das Bild erstellt wurde) zu 
speichern. Da die allermeisten Digitalkameras (noch) keinen GPS-Empfänger haben, müssen die Bilder nachträglich mit 
den Geoinformationen versehen werden. Dazu nimmt man die Uhrzeit, an der ein Bild aufgenommen wurde und sucht aus dem 
passenden GPX-Track die zugehörige Koordinate. Damit das zuverlässig klappt, muss man die Uhrzeit der Kamera, mit der 
GPS-Zeit abgleichen. \\
Dazu fotografiert man \textbf{während} der Tour das Display des GPS-Gerätes mit der eingeblendeten Uhrzeit. Später 
kann man in JOSM dann beim Importieren der Bilder die Uhrzeit von dem Bild eingeben. JOSM übernimmt dann die Zuordnung
der Fotos zu den jeweiligen Koordinaten. \\

Tipp: Prune kann .gpx Tracks beschneiden. 


\newpage
\section*{Wichtige Tags}

\section*{Besonders wichtige Plugins}
\begin{description}
\item[Openstreetbugs] Kartenfehler markieren und die Markierungen bearbeiten
\item[terracer] Hausbautool für Reihenhäuser
\item[validator] Prüft Karte auf logische Unstimmigkeiten
(Achtung! Erzeugt manchmal Fehlalarme!) 
\item[wms] Zeigt Luftbilder im Hintergrund an. Manchmal sehr kompliziert in der Installation, langsamer Bildaufbau. 
Anleitung im OSM-Wiki beachten.
\item[editgpx] Ermöglicht das Löschen von Punkten aus GPX-Tracks direkt in JOSM (z.B. um Punktewolken zu entfernen)
\end{description}

\newpage
Diese Befehlsreferenz wurde von Jonas Stein in \LaTeX gesetzt.
Verbesserungsvorschläge bitte an news@jonasstein.de mailen.
Die Quelldaten sind online erreichbar unter: \texttt{http://github.com/jonasstein/josmrefcard/}

\textbf{Dank an:} % Bitte hier mit Komma separiert nach einer Ergaenzung selbst eintragen.


\end{document}


%%% Local Variables: 
%%% mode: latex
%%% TeX-master: t
%%% End: 
